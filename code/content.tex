\section{Einführung}

Kanji (漢字) sind möglich.
Weitere Sprachen sind auch verfügbar, siehe dazu das CJK Paket \url{https://ctan.org/pkg/cjk}.
Sowie Refernzen auf Abbildungen \ref{fig:Diebstahl in der Fischerei}. Wenn man die Zahl anklickt, wird man zur Abbildung geleitet.
\section{Weitere Beispiele}
textsc wird für \textsc{Kapitälchen} genutzt.
textsl für \textsl{geneigten Text}.
\\
\begin{addmargin}[50pt]{50pt}Bei längeren Zitaten muss man das Zitat als Block einrücken.
Das macht man mit \textbackslash begin\{addmargin\}[50pt]\{50pt\} und end\{addmargin\}.
Vergesst aber bei Befehlen nicht das Blackslash vor dem jeweiligen Befehl.\\  
\end{addmargin}

\enquote{Für Zitate wird enqoute benutzt}.

\subsection{\textsl{Unterkapitel}} 

Normale Fußnoten sind möglich\footnote{Fußnote}.
Möchte man aber auf eine Fußnote später erneut verweisen, kann man footnoteremember benutzen\footnoteremember{label123}{Dies ist eine Fußnote auf die erneut verwiesen werden kann.
Sie wird normal gezählt wie die anderen Fußnoten auch.
Einer footnoteremember muss man ein Label geben, damit sie von anderen footnoteremeber unterschieden werden kann.}.
\newpage
Auf dieser Seite können wir auf die Fußnote mit footnoterecall erneut verweisen \footnoterecall{label123}.
Mit \mbox{newpage} kann man eine neue Seite befehlen. Mit mbox kann man falsch getrennte Wörter wieder zusammenschreiben.
